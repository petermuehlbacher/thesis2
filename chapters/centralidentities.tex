To prove our main results we first need some refinements for the expected values of critical values.

\begin{theorem}[Express ${\mathbb E[Crt_{N,k}(B)]}$ in terms of the GOE]
	For all $B$ Borel sets, $N$, $p\geq 2$ and $k\in\{0,\dots,N-1\}$ we have

	\begin{equation}\label{thm:2.1}
		\mathbb E[Crt_{N,k}(B)]=2\sqrt{\frac{2}{p}}(p-1)^{\frac{N}{2}}\mathbb E_{GOE}^N\left[e^{-N\frac{p-2}{2p}(\lambda_k^N)^2}\bm 1\left\{\lambda_k^N\in\sqrt{\frac{p}{2(p-1)}}B \right\}\right]
	\end{equation} and
	
	\begin{equation}\label{thm:2.2}
		\mathbb E[Crt_N(B)]=2N\sqrt{\frac{2}{p}}(p-1)^{\frac{N}{2}}\int_{\sqrt{\frac{p}{2(p-1)}}B}exp\left\{-\frac{N(p-2)x^2}{2p}\right\}\rho_N(x)dx.
	\end{equation}
\end{theorem}

\begin{proof}
First of all let us show that we can apply lemma \ref{thm:KacRice}. To do so we use the same notation as in section \ref{sec:GRF}. Because of the rotational symmetry and since $S^{N-1}$ can be covered by a finite number of copies of an open neighbourhood of some point it suffices to investigate only one point. We will choose the north-pole $n$ and see that $(f_i(n),f_{ij}(n))$ is not degenerate. Since the covariances are continuous this is also true for some neighbourhood $U$. Hence the conditions of lemma \ref{thm:KacRice} are satisfied. Again due to the rotational symmetry the integrand does not depend on $\sigma$, so we get

\begin{align*}
	&\mathbb E Crt_{N,k}(B)=\\
	&\text{vol}(S^{N-1})\mathbb E[|\det\nabla^2 f(n)|\mathbf 1\{i(\nabla^2 f(\sigma))=k, f(n)\in\sqrt NB\}|\nabla f(n)=0]d\mathbb P(\nabla f(n)=0)
\end{align*}

To compute this expectation we condition on $f(n)$
\begin{align*}
	&\mathbb E[|\det\nabla^2 f(n)|\mathbf 1\{i(\nabla^2 f(\sigma))=k, f(n)\in\sqrt NB\}|\nabla f(n)=0]=\\
	&\mathbb E\left[\mathbb E[|\det\nabla^2 f(n)|\mathbf 1\{i(\nabla^2 f(\sigma))=k,f(n)\in\sqrt NB\}|f(n)]\right]
\end{align*}

By lemma the following equality for the interior expectation

\begin{align*}
	&\mathbb E[|\det\nabla^2 f(n)|\mathbf 1\{i(\nabla^2 f(\sigma))=k,f(n)\in\sqrt NB\}|f(n)]=\\
	&(2(N-1)p(p-1))^{(N-1)/2}\mathbb E_{GOE}^{N-1}\Big[|\det(M^{N-1}-\sqrt{p/(2(N-1)(p-1))}f(n)I)|\\&\times\mathbf 1\{i(M^{N-1}-\sqrt{p/(2(N-1)(p-1))}f(n)I)=k,f(n)\in\sqrt NB\}\Big].
\end{align*}

Substituting back, writing $t^2$ for $p/(2(N-1)(p-1))$ as well as $G$ for $\sqrt{Np/(2(N-1)(p-1))}B$ and setting $X$ to be a real valued Gaussian random variable with zero mean and variance $t^2$ we get 

\begin{align}\label{eq:expressEintermsofEGOE}
	&\mathbb E\Big[|\det(M^{N-1}-XI)\mathbf 1\{i(M^{N-1}-XI)=k,X\in G\}\Big]=\nonumber\\
	&(2\pi t^2)^{-1/2}\int_G\exp(-\frac{x^2}{2t^2})\mathbb E_{GOE}^{N-1}\Big[|\det(M-xI)|\mathbf 1\{i(M-xI)=k\}\Big]\diff x.
\end{align}

Now observe that the event $\{i(M-xI)=k\}$ is equal to the event $\{A_k^N(x)\}$, where $A_k^N(x) = \{\lambda^{N-1}: \lambda_0^{N-1}\leq\dots\leq\lambda_{k-1}^{N-1}<x\leq\lambda_k^{N-1}\leq\dots\leq\lambda_{N-2}^{N-1}\}$.

Using \ref{thm:probabilitydensityofEV} we get

\begin{equation*}
	\mathbb E_{GOE}^{N-1}\left[|\det(M-xI)|\mathbf 1\{i(M-xI)=k\}\right]=\int_{A_k^N(x)}\prod_{i=1}^{N-2}|\lambda_i^{N-1}-x|Q_{N-1}(\diff\lambda^{N-1}).
\end{equation*}

The definition of $A_k^N(x)$ and the determinant in the equation above suggests considering $x$ as the $k+1$-th smallest eigenvalue of a $N\times N$ GOE matrix. Performing the corresponding rescaling, that is, change of variables given by $\lambda_i^{N-1}=\sqrt{N/(N-1)}\lambda_i^N$ for $i\in\{0,\dots,k-1\}$, $\lambda_i^{N-1}=\sqrt{N(/(N-1)}\lambda_{i+1}^N$ for $i\in\{k,\dots,N-2\}$ and $x=\sqrt{N/(N-1)\lambda_k^N}$, writing out $Q_N$ and substituting back in equation \ref{eq:expressEintermsofEGOE} we obtain

\begin{align*}
	&\frac{Z_N}{Z_{N-1}\sqrt{2\pi t^2}}(N/(N-1))^{(N+2)(N+1)/4}\\
	&\times\mathbb E_{GOE}^N\Big[\exp\Big(\frac{N(\lambda_k^N)^2}{2}-\frac{\frac{N}{N-1}(\lambda_k^N)^2}{2t^2}\Big)\mathbf 1\{\lambda_k^N\in\sqrt{(N-1)/N)}G\}\Big],
\end{align*}

where the constants $Z_N$ is given by $$\frac{1}{N!}(2\sqrt 2)^N N^{-N(N+1)/4}\prod_{i=1}^N \Gamma(1+i/2)$$ which can be computed from Selberg's integral (cf. \cite{Mehta2004random}).

Some straightforward algebra then yields the first claim.

The second one is obtained by summing over $k\in\{0,\dots,N-1\}$.

%\frac{\Gamma(N/2)(N-1)^{-N/2}}{\sqrt{\pi t^2}}\mathbb E_{GOE}^{N}\Bigg[\exp\Big(\frac{N(\lambda_k^N)^2}{2}-\frac{\frac{N}{N-1}(\lambda_k^N)^2}{2t^2}\Big)\mathbf 1\{\lambda_k^N\in\sqrt{\frac{N-1}{N}}G\}\Bigg]
\end{proof}

Now we are ready to give a sketch of the proof of theorem \ref{thm:2.5}:

\begin{proof}
	For simplicity's sake we fix $B$ in $\mathbb E Crt_{N,k}(B)$ to be $(-\infty,u)$. The proof naturally extends to arbitrary Borel sets $B$.
	
	To declutter notation we set $t=u\sqrt{\frac{p}{2(p-1)}}$, $\phi(x)=-\frac{p-2}{2p}x^2$ and $J_k(u)=(k+1)I_1(-u;2^{-1/2})$, where the additional argument $2^{-1/2}$ indicates some normalisation factor that we get since we are not dealing with random variables with variance one in this setting.
	
	Now by theorem \ref{thm:2.1},
	
	$$\lim_{N\rightarrow\infty}\frac{1}{N}\log\mathbb E Crt_{N,k}(u) = 
	\frac{1}{2}\log(p-1) + \lim_{N\rightarrow\infty}\frac{1}{N}\log\mathbb E_{GOE}^N[\exp(N\phi(\lambda_k^2)\mathbf 1_{\lambda_k\leq t}].$$
	
	Since $\phi$ is bounded from above we can apply Varadhan's lemma to obtain
	
	$$\sup_{x\in(-\infty,t)}(\phi(x)-J_k(x))\leq\liminf_{N\rightarrow\infty}\frac{1}{N}\log\mathbb E_{GOE}^N[\exp(N\phi(\lambda_k^2)\mathbf 1_{\lambda_k<t}]$$
	$$\leq\limsup_{N\rightarrow\infty}\frac{1}{N}\log\mathbb E_{GOE}^N[\exp(N\phi(\lambda_k^2)\mathbf 1_{\lambda_k\leq t}]\leq\sup_{x\in(-\infty,t]}(\phi(x)-J_k(x)).$$
	
	For $t\leq-\sqrt 2$ both suprema equal $\phi(t)-J_k(t)$ and if $t>-\sqrt 2$ they equal $\phi(\sqrt 2)$. The claim essentially follows after appropriate rescaling of the rate function $I_1$ (cf. \cite{Cerny10}).
\end{proof}

Applying Markov's inequality to theorem \ref{thm:2.5} (as in \cite{Cerny10} proof of theorem 2.14) then yields \ref{thm:2.15}.
