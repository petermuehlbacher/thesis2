\section{Motivation}
Given some Hamiltonian $H_{N,p}: S^{N-1}\rightarrow\mathbb R$ given by $$H_{N,p}(\bm\sigma)=\frac{1}{N^{(p-1)/2}}\sum_{i_1,\dots,i_p=1}^N J_{i_1,\dots,i_p}\sigma_{i_1}\dots\sigma_{i_p},$$ where the $J_{i_1,\dots,i_p}$ are independent centered standard Gaussian random variables and $\bm\sigma=(\sigma_1,\dots,\sigma_N)$ referred to as states, one is interested in finding the expected number of critical points $Crt_N(u)$ and the restriction to some index $k$, denoted by $Crt_{N,k}(u)$, in some region in $(-\infty,Nu]$.

For practical applications like neural networks it may be nice to know how the critical points are distributed, but there is still the open question whether local minima that are close to the global minimum will also yield ``good results''. From a practical point of view this has been empirically confirmed (\cite{LeCun2014loss}), however, it can also be shown theoretically like in \cite{loh2013regularized} for a far more general class of problems.

\section{Main Results}
\begin{theorem}[Large deviations for ${\mathbb E[\text{Crt}_{N,k}(u)]}$]
	\begin{equation}\label{thm:2.5}
		\lim_{N\rightarrow\infty}\frac{1}{N}\log\mathbb E[\text{Crt}_{N,k}(u)] = \Theta_{k,p}(u)
	\end{equation}
	and
	\begin{equation}\label{thm:2.8}
		\lim_{N\rightarrow\infty}\frac{1}{N}\log\mathbb E[\text{Crt}_N(u)] = \Theta_p(u),
	\end{equation}
	for $$\Theta_p(u)=\begin{cases}
						\frac{1}{2}\log(p-1)-\frac{p-2}{4(p-1)}u^2-I_1(u), &\mbox{if } u\leq-E_\infty \\
						\frac{1}{2}\log(p-1)-\frac{p-2}{4(p-1)}u^2, &\mbox{if } -E_\infty\leq u\leq 0\\ 
						\frac{1}{2}\log(p-1), &\mbox{if } 0\leq u
					  \end{cases}$$
	and
	$$\Theta_{k,p}(u)=\begin{cases}
						\frac{1}{2}\log(p-1)-\frac{p-2}{4(p-1)}u^2-(k+1)I_1(u), &\mbox{if } u\leq-E_\infty \\
						\frac{1}{2}\log(p-1)-\frac{p-2}{p}, &\mbox{if } u \geq E_\infty
					  \end{cases},$$
	where $E_\infty=E_\infty(p)=2\sqrt{\frac{p-1}{p}}$ and $I_1:(-\infty,E_\infty]\rightarrow\mathbb R$ is given by $$I_1(u)=\frac{2}{E_\infty^2}\int_u^{-E_\infty}\sqrt{z^2-E_\infty^2}dz$$ and is the the rate function of the LDP for the smallest eigenvalue of the GOE.
\end{theorem}

\begin{theorem}[Layered structure]
	For all $k\geq 0$ and $\varepsilon>0$ we have
	\begin{equation}\label{thm:2.15}
		\limsup_{N\rightarrow\infty}\frac{1}{N}\log\mathbb P\left(\left\{\sum_{i=k}^\infty Crt_{N,i}(-E_k-\varepsilon)>0\right\}\right)<0,
	\end{equation}
	where $E_k=E_k(p)$ is chosen (uniquely because of strict monotonicity) such that $\Theta_{k,p}(-E_k)=0$.
\end{theorem}

\section{Outlook}
According to \cite{LeCun2014lossvariance} there is empirical evidence that the energy of critical points concentrates around the limiting floor value, however, up to now and to the best of my knowledge there is no proof for this theoretical statement and in particular (from a theoretical point of view) nothing is known about the speed of convergence to the expected value that is presented here.












