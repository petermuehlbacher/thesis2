To be able to show the previously mentioned results we first need to prove some other theorems which require prerequisites from various other fields.

\section{GOE}
\begin{definition}\label{def:GOE}
	Symmetric $N\times N$ matrices $H=H_N$ with $\mathbb E H_{ij}=0$ and $\mathbb E H_{ij}^2=1+\delta_{ij}$.
\end{definition}
\begin{remark}[Density in the space of matrices]
	Their density is given by the Gaussian measure
	\begin{equation}
		d\mathbb P(H)=Z_N^{-1}\exp\left(-N\frac{1}{4}tr H^2\right)
	\end{equation}
	with normalization constant $Z_N=\int d\mathbb P(H)\prod_{1\leq i\leq j\leq N} dH_{ij}$ which is a shorter way of writing $Z_N^{-1}\exp\left\{-N\frac{1}{4}\left(2\sum_{i<j}^N H_{ij}^2+\sum_{i=j}^N H_{ij}^2\right)\right\}$.
\end{remark}

\begin{theorem}[Joint probability density of eigenvalues]\label{thm:probabilitydensityofEV}
	The joint probability density $Q_N$ of the unordered eigenvalues $\{\lambda_i\}_{i=1}^N$ of the GOE is given by
	$$Q_N(\diff\lambda_1,\dots,\diff\lambda_N) = C_N\prod_{i<j}|\lambda_i-\lambda_j|\prod_i\exp(-N\lambda_i^2/4)\diff\lambda_i,$$
	
	where, without loss of generality, we assumed the variances of the entries to be normalized to one and for some $C_N>0$ which is uniquely determined by normalization.
\end{theorem}
\begin{proof}
	This proof will follow the one given in \cite{LiuEigenvalues}.
	
	Let $H$ be some element of the GOE. Since it is symmetric there is a decomposition in $H=UDU^T$ with $U$ orthogonal and $D=\text{diag}(\lambda_1,\dots,\lambda_N)$. Thus we can write 
	\begin{equation}\label{eq:H_ijLinearityInEigenvalues}
		H_{ij} = \sum_k \lambda_k U_{ik}U_{jk}
	\end{equation}
	and, using the orthogonality of $U$: $\sum_k U_{ki}U_{kj}=\delta_{ij}$. Using those results one can infer that $\sum_{i,j}H_{ij}^2=\sum_i \lambda_i^2$.
	
	The key idea is to make use of the change of variables formula to go from $\mathbb P(\lambda_1,\dots,\lambda_N,\alpha_1,\dots,\alpha_{N(N-1)/2})$ to $\mathbb P(H)$. One should think of the $\{\alpha_i\}_{i=1}^{N(N-1)/2}$ as the parameters that determine the matrix $U$, which, together with the eigenvalues, uniquely determines $H$.
	
	To do that we first need some information on the determinant of the Jacobian $J$ of the change of variables:
	From equation \ref{eq:H_ijLinearityInEigenvalues} we see the linearity of $H_{ij}$ in the eigenvalues $\lambda_k$ which implies that $\partial H_{ij}/\partial\alpha$ is linear in the eigenvalues\footnote{To be precise, it is linear in the \textit{vector} $(\lambda_i)_{i=1}^N$.}. Hence, $\det J$ has to be a polynomial of degree $N(N-1)/2$ in the eigenvalues. If two eigenvalues coincide $U$ cannot be uniquely determined anymore and thus the inverse of the transformation is not unique, meaning that $\det J=0$. So the determinant of the Jacobian must vanish for all $\lambda_i=\lambda_j, i\neq j$, which is achieved if it contains a factor $\lambda_i-\lambda_j$. However, there are exactly $N(N-1)/2$ such factors and since that is just the degree of the polynomial it follows that we have completely accounted for $J$'s dependence on the eigenvalues by writing $$\det J=\prod_{i<j}(\lambda_i-\lambda_j) h(\alpha_1,\dots,\alpha_{N(N-1)/2}).$$
	
	Now we can write
	$$\mathbb P(\lambda_1,\dots,\lambda_N,\alpha_1,\dots,\alpha_{N(N-1)/2}) = $$
	$$\mathbb P(H)|\det J|=Z_N^{-1}\exp\left(-N\frac{1}{4}\sum_i\lambda_i^2\right)|\prod_{i<j}(\lambda_i-\lambda_j) h(\alpha_1,\dots,\alpha_{N(N-1)/2})|.$$
	
	Integrating out the dependence on $\{\alpha_k\}_{k=1}^{N(N-1)/2}$ yields the desired result.
\end{proof}


\section{Gaussian Random Fields}\label{sec:GRF}
For what follows it we shall fix some notations like in \cite{Cerny10}:

First of all, instead of working with $H_{N,p}$ it is convenient to define

\begin{equation}\label{def:rescaledHamiltonian}
	f(\sigma)\equiv f_{N,p}(\sigma)=N^{-1/2}H_{N,p}(N^{1/2}\sigma),
\end{equation}


which has variance one on the unit sphere. We use $\langle\cdot,\cdot\rangle$ to denote the usual Euclidean scalar product, as well as the scalar product on any tangent space $T_\sigma S^{N-1}$ and $(E_i)_{1\leq i<N}$ to denote an arbitrary orthonormal frame field, that is a set of $N-1$ vector fields $E_i$ on $S^{N-1}$ such that $\{E_i(\sigma)\}$ is an orthonormal basis of $T_\sigma S^{N-1}$. Furthermore we write $\nabla f(\sigma)=(f_i(\sigma))_{1\leq i<N}$ for the gradient $(E_i f(\sigma))_{1\leq i<N}$, $\nabla^2 f = (f_{ij})_{1\leq i,j<N}$ for the covariant Hessian of $f$ on $S^{N-1}$ and $\det\nabla^2 f(\sigma)$ for the determinant of the matrix $(\nabla^2 f(E_i,E_j)(\sigma))_{1\leq i,j<N}$.

\begin{definition}[Real valued random fields]
	Let $(\Omega,\mathcal F, \mathbb P)$ be a complete probability space and $T$ a topological space. Then a measurable mapping $f:\Omega\rightarrow\mathbb R^T$ is called a real valued random field.
	
	In what follows we will not distinguish between $f(\omega)(t)$ and $f(t)$.
\end{definition}
In other words, a random field $f$ is a collection $\{f(t):t\in T\}$ where every $f(t)$ is a real valued random variable.

\begin{definition}[Real valued Gaussian fields]
	Let $f$ be a real valued random field with some parameter set $T$ as before. Then $f$ is a real valued Gaussian field if for every $(t_1,\dots,t_n)\in T^n$ ($1\leq n <\infty$) the distributions $(f(t_1),\dots,f(t_n))$ are multivariate Gaussian.
	
	Furthermore the functions $m(t)=\mathbb E[f(t)]$ and $C(t,t')=\mathbb E[(f(t)-m(t))(f(t')-m(t'))]$ are called the mean and covariance functions of $f$, respectively.
\end{definition}
Note that this definitions yields the usual definition for a multivariate Gaussian $\mathbb R^d$-valued random variable in case $|T|=d$ is finite.

The formulation and proofs of the following two lemmata will closely follow \cite{Cerny10} (and \cite{Adler07} for the proofs).

\begin{lemma}[Covariances of $f$]\label{thm:covarianceStructure}
	Let $f$ be as defined in \ref{def:rescaledHamiltonian}, then $f(\sigma),f_i(\sigma),f_{ij}(\sigma)$ are centered Gaussian random variables for all $1\leq i,j,k,l<N$ and $\sigma\in S^{N-1}$ whose joint distribution is determined by
	
	\begin{align}\label{eq:covariancesoff}
		\mathbb E[f(\sigma)^2]&=1,\nonumber\\
		\mathbb E[f(\sigma)f_i(\sigma)]=\mathbb E[f_i(\sigma)f_{jk}(\sigma)]&=0,\nonumber\\
		\mathbb E[f_i(\sigma)f_j(\sigma)]=-\mathbb E[f(\sigma)f_{ij}(\sigma)] &= p\delta_{ij},\nonumber\\
		\mathbb E[f_{ij}(\sigma)f_{kl}(\sigma)] &= p(p-1)(\delta_{ik}\delta_{jl}+\delta_{il}\delta_{jk})+p^2\delta_{ij}\delta_{kl}.
	\end{align}
\end{lemma}

\begin{proof}
	Because of the rotational symmetry we may assume without loss of generality that $\sigma$ is the north pole $n=(0,\dots,0,1)$. We define the projection $\Psi:S^{N-1}\rightarrow\mathbb R^{N-1}, (x_1,\dots,x_{N-1},x_N)\mapsto(x_1,\dots,x_{N-1})$ which is a chart for some neighbourhood $U$ of $n$ and set $\overline f = f\circ\Psi^{-1}$ which is a Gaussian process on $\Psi(U)$, satisfying $$C(x,y)=Cov(\overline f(x),\overline f(y))=\left(\sum_{i=1}^{N-1}x_iy_i + \sqrt{(1-\textstyle\sum_{i=1}^{N-1}x_i^2)(1-\textstyle\sum_{i=1}^{N-1}y_i^2)}\right)^p.$$
	
	Choosing an orthonormal frame field $(E_i)$ such that it satisfies $E_i(n)=\partial/\partial x_i$ with respect to the chart $\Psi$ the Christoffel symbols vanish and hence the covariant Hessian $(f_{ij}(n))$ agrees with the usual Hessian of $\overline f$ at $0$.
	
	Thus it suffices to prove the analogous identities for $\overline f(0),\frac{\partial}{\partial x_i}\overline f(0),\frac{\partial^2}{\partial x_i\partial x_j}\overline f(0)$.
	Differentiation under the integral sign yields 
	\begin{equation}
	\mathbb E\left\{\frac{\partial^k \overline f(s)}{\partial s_{i_1}\dots\partial s_{i_k}} \frac{\partial^l \overline f(t)}{\partial t_{i_1}\dots\partial t_{i_l}}\right\}=\frac{\partial^{k+l} C(s,t)}{\partial s_{i_1}\dots\partial s_{i_k}\partial t_{i_1}\dots\partial t_{i_l}},
	\end{equation}
	which, in turn, after some simple algebra gives \ref{eq:covariancesoff}.
\end{proof}

Now we can formulate one of the central identities that allow us to use methods from random matrix theory to study the Hamiltonian of interest.

\begin{lemma}[Moments under conditioning]
	Using the same assumptions as in the previous lemma and under the conditional distribution $\mathbb P[\cdot|f(\sigma)=x], x\in\mathbb R$ the random variables $f_{ij}(\sigma)$ are independent Gaussian variables satisfying
	\begin{align}\label{eq:momentsUnderConditioning}
		\mathbb E[f_{ij}(\sigma)|f(\sigma)=x]&=-xp\delta_{ij},\nonumber\\
		\mathbb E[f_{ij}(\sigma)^2|f(\sigma)=x]&=(1+\delta_{ij})p(p-1).
	\end{align}
	
	Alternatively, the random matrix $(f_{ij}(\sigma))$ (under the conditional distribution $\mathbb P[\cdot|f(\sigma)=x], x\in\mathbb R$) has the same distribution as $$M^{N-1}\sqrt{2(N-1)p(p-1)}-xpI,$$
	
	where $M^{N-1}$ is a $(N-1)\times(N-1)$ GOE matrix as defined in \ref{def:GOE} and $I$ is the identity matrix.
\end{lemma}
\begin{proof}
	Equations \ref{eq:momentsUnderConditioning} can be seen from the last two equations of \ref{eq:covariancesoff} and by using $\mu_{Y|X} = \mu_Y + \Sigma_{YX}\Sigma^{-1}_{XX}(x-\mu_X)$ for $(X,Y)\sim\mathcal N(\mu;\Sigma)$ and setting $X$ to be $f(\sigma)$ and $Y$ to be $f_{ij}(\sigma)$.
	
	The second statement follows by plugging in the definitions.
\end{proof}




\section{LDP}
Note that some theorems in this section will not be proven because their proofs are rather technical in nature and do not contribute very much to a better understanding of the subject.

\begin{definition}[Large deviation principle]\label{def:LDP}
	Given some separable completely metrizable topological space\footnote{Henceforth such a space will be referred to as Polish space.} $X$, a sequence of Borel probability measures $\{\mathbb P_n\}$ on $X$ is said to satisfy a \textbf{large deviation principle} with speed $\{a_n\}$ and rate $I:X\rightarrow [0,\infty]$ if $a_n$ goes to $+\infty$ and $I$ is some lower semi-continuous functional such that for each Borel measurable set $E\subseteq X$ we have
	
	$$\limsup_n a_n^{-1}\log(\mathbb P_n(E))\leq -\inf_{x\in \overline E}I(x)$$
	and
	$$\liminf_n a_n^{-1}\log(\mathbb P_n(E))\geq -\inf_{x\in E^\circ}I(x).$$
	
	The lower semi-continuity implies that the sets $\{x\in X: I(x)\leq c\}$ are closed in $X$ for all $c\geq 0$. If they are also compact for all $c\geq 0$, $I$ is called a \textbf{good rate function}.
\end{definition}

The following definition and theorem are taken from \cite{Dembo2009LargeDeviations}:
% https://books.google.at/books?id=iT9JRlGPx5gC&pg=PA130&dq=large+deviations+exponential+equivalent&hl=en&sa=X&ved=0ahUKEwj9zorh5MDKAhXns3IKHY12A9wQ6AEIHzAA#v=onepage&q&f=false

\begin{definition}[Exponential equivalence]\label{def:exponentialequivalence}
	Two families of probability measures $\{\mu_\varepsilon\}, \{\tilde \mu_\varepsilon\}$ on some metric space $(Y,d)$ are called exponentially equivalent if there exist probability spaces $\{\Omega, \mathcal B_\varepsilon,\mathbb P_\varepsilon\}$ and two families of $Y$-valued random variables $\{Z_\varepsilon\}, \{\tilde Z_\varepsilon\}$ with joint laws $\{\mathbb P_\varepsilon\}$ and marginals $\{\mu_\varepsilon\}, \{\tilde \mu_\varepsilon\}$, respectively, such that the following holds:
	
	For each $\delta>0$, the set $\{\omega : (Z_\varepsilon, \tilde Z_\varepsilon)\in\Gamma_\delta\}$ is $\mathcal B_\varepsilon$ measurable, and 
	
	$$\limsup_{\varepsilon\rightarrow 0}\varepsilon\log\mathbb P_\varepsilon(\Gamma_\delta)=-\infty,$$
	
	where $\Gamma_\delta = \{(y,\tilde y):d(y,\tilde y)>\delta\}\subseteq Y\times Y$.
\end{definition}

For instance, if the random variables are real valued and the rate of the LDP is $N$ (i.e. $\varepsilon = N^{-1}$), this asserts that the joint probability $\mathbb P_N$ of the two random variables of the area that is farther than $\delta$ away from the diagonal $x=y$ goes to zero like some $\exp(-cN^p)$ for some constants $c>0, p>1$. A quick sketch illustrates that the two probability measures have to be ``quite'' similar to satisfy the above definition -- this can be made precise by the following theorem:

\begin{theorem}[Rate functions of exponentially equivalent measures]
	If a LDP with good rate function $I$ holds for the probability measures $\{\mu_\varepsilon\}$, which are exponentially equivalent to $\{\tilde\mu_\varepsilon\}$, then the same LDP holds for $\{\tilde\mu_\varepsilon\}$.
\end{theorem}
\begin{proof}
	In the general case this is a very technical result and hence we will refer the interested reader to \cite{Dembo2009LargeDeviations}, theorem 4.2.13.
\end{proof}

We are particularly interested in getting LDPs of the $k$-th largest eigenvalues of the GOE, but it turns out in order to get those we first need a LDP for the law of the empirical measure $\mu^N=\frac{1}{N}\sum_{i=1}^N\delta_{\lambda_i}$.

\begin{theorem}[LDP of Wigner's semicircle law for the GOE]\label{thm:LDPforWSCL}
	Let $I(\mu):=\frac{1}{2}\left(\int x^2\diff\mu(x) - \Sigma(\mu) - \frac{3}{4}-\frac{1}{2}\log 2 \right)$, for $\Sigma(\mu):=\int\int\log|x-y|\diff\mu(x)\diff\mu(y)$. Then:
	\begin{enumerate}
		\item $I$ is well defined on the space of probability measures $\mathcal M(\mathbb{R})$ on $\mathbb{R}$ endowed with the weak topology and takes values in $[0,\infty]$.
		\item $I(\mu)=\infty \Leftrightarrow \int x^2\diff\mu(x)=\infty$ or $\exists A\subseteq\mathbb R: \mu(A)>0 \land \exp\{-\inf_{\nu\in\mathcal M(A)}\int\int \log|x-y|^{-1}\diff\mu(x)\diff\mu(y)\}=0$, where the latter condition is usually referred to as the existence of a set of ``null logarithmic capacity''.
		\item $I$ is a good rate function.
		\item $I$ is a convex function.
		\item $I$ achieves its unique minimum value at Wigner's semicircle law.
	\end{enumerate}
	In particular the law of the empirical measure $\mu^N$ satisfies a LDP with good rate function $I$ and speed $N^2$.
\end{theorem}

A proof can be found in \cite{ArousLDPforWSL}.
%\begin{proof}
%	We will only give a very rough outline of the proof that can be found in . \todo{give rough outline}
	
	%To prove that $I$ is we	ll defined one splits it up in $I(\mu)=H(\mu)+C$. 
%\end{proof}

With this we can state the main result of this section. Note that without loss of generality we set the variances of the entries of the GOE to $1$.
\begin{theorem}[LDP for the $k$-th largest eigenvalue of the GOE]\label{thm:A1}
	For each fixed $k\geq 1$, the $k$-th largest eigenvalue $\lambda_{N-k+1}$ of the GOE satisfies a LDP with speed $N$ and good rate function
	$$I_k(x)=kI_1(x)=\begin{cases}
						k\int_2^x\sqrt{\frac{z^2}{4}-1}\diff z, &\mbox{if } x\geq 2 \\
						\infty, &\mbox{else.}
					  \end{cases}$$
\end{theorem}
\begin{proof}
	We will prove the two cases $x<2$ and $x\geq 2$ separately and start with the first one:
	
	Since $\lambda_{N-k+1}\leq x<2$ we have that the empirical spectral measure $\mu_N((x,2])\leq \frac{k-1}{N}$. However, Wigner's semicircle law implies that $\mu_{sc}((x,2])=\lim_{N\rightarrow\infty} \mu_N((x,2])>0$. So there exists a closed set $A\in\mathcal P(\mathbb R)$ such that $\mu_{sc}\notin A$, but the set of all empirical spectral distributions with $k$-th largest eigenvalue being smaller than $x$ being a subset of $A$, i.e. $\{\tilde \mu_N:\lambda_{N-k+1}\leq x\}\subseteq A$. Because of the LDP with speed  $N^2$ of $\{\mu_N\}_N$ we have $Q_N(A)\leq\exp(-cN^2)$ for some $c>0$ which concludes the case for $x<2$.
	
	The second part will be split up in showing the upper and lower bounds of the equation. We will start with the upper one
	
	\begin{equation}\label{eq:LDPlargestEVupperbound}
		\limsup \frac{1}{N}\log Q_N(\lambda_{N-k+1}\geq x)\leq -I_k(x).
	\end{equation}
	
	However, instead of showing this directly we will use the obvious\footnote{This is basically $\mathbb P(E)=\mathbb P(\neg F)+\mathbb P(E\land F)$.} upper bound
	
	$$Q_N(\lambda_{N-k+1}\geq x)\leq Q_N(\max_{i=1}^N |\lambda_i|\geq M)+ Q_N(\lambda_{N-k+1}\geq x,\max_{i=1}^N|\lambda_i|<M),$$
	
	which, together with the estimate (see \cite{ArousAging})
	
	\begin{equation}\label{eq:maxEVinequality}
		Q_N(\max_{i=1}^N|\lambda_i|\geq M)\leq \exp(-NM^2/9)
	\end{equation}
	
	for $M$ large enough and all $N$\footnote{It follows by integrating the elementary estimate $|x-\lambda_i|e^{-\lambda_i^2/4}\leq (|x|+|\lambda_i|)e^{-\lambda_i^2/4}\leq 2|x|\leq e^{x^2/8}$ which holds for $|x|\geq M \geq 8$ and using the fact that $Z_{N-1}/Z_N\leq e^{CN}$ for some $C$ independent of $N$ (see Selberg's formula in \cite{Mehta2004random}).} lets us prove \ref{eq:LDPlargestEVupperbound} by showing
	
	$$\limsup\frac{1}{N}\log Q_N(\max_{i=1}^N|\lambda_i|\leq M, \lambda_{N-k+1}\geq x)\leq-I_k(x),$$
	
	for all $M>x>2$.
	
	To that end let $\overline Q_{N-k}^N$ be a rescaled version of the joint law of unordered eigenvalues $\overline Q_{N-k}$, given by 
	$$\overline Q_{N-k}^N(\lambda\in\cdot) = \overline Q_{N-k}(\sqrt{1-k/N}\lambda\in\cdot),$$ where the factor $\sqrt{1-k/N}=\sqrt{\frac{N-k}{N}}$ serves the purpose of ``changing the units'' of an eigenvalue $\lambda$ from an $(N-k)$-dimensional distribution (which is normalised by $(N-k)^{-1/2}$) to an $N$-dimensional one.
	
	Similar considerations apply for the new normalisation constant 
	
	$$C_N^k = (1-k/N)^{(N-k)(N-k+1)/4}\frac{\overline Z_{N-k}}{\overline Z_N},$$
	
	where $\overline Z_N$ is the normalization factor for $\overline Q_N$.
	
	For $x\in\mathbb R$ and $\mu\in\mathcal P(\mathbb R)$ we define
	
	$$\Phi(x,\mu)=\int_\mathbb{R}\log|x-y|\diff\mu(y)-\frac{x^2}{4},$$
	
	which can be shown (\cite{ArousAging}, p.50) to be upper semi-continuous on $[-M,M]\times\mathcal P([-M,M])$ and continuous on $[x,y]\times\mathcal P([-M,M])$ for $x,y,M$ such that $2<M<x<y$.
	
	Using \ref{thm:probabilitydensityofEV} we see that
	
	\begin{align*}
		Q_N(\max_{i=1}^N&|\lambda_i|\leq M,\lambda_{N-k+1}\geq x)= \\ N! Z_N^{-1}&\int_{[x,M]^k}\int_{[-M,M]^{N-k}}\left(\prod_{1\leq i<j\leq N}|\lambda_i-\lambda_j|\prod_{i=1}^N \exp(-N\lambda_i^2/4)\right)\\
		&\diff\lambda_1\dots\diff\lambda_{N-k}\diff\lambda_{N-k+1}\dots\diff\lambda_N,
	\end{align*}
	
	where the $N!$ comes from dropping the $\mathbf 1_{\lambda_1\leq\dots\lambda_N}$ term. This can be bounded from above by
	
	\begin{align}\label{eq:upperbound1forQN}
		C_N^k\frac{N!}{(N-k)!}&\int_{[x,M]^k}\prod_{N-k<i<j\leq N}|\lambda_i-\lambda_j| \int_{[-M,M]^{N-k}}e^{(N-k)\sum_{i=N-k+1}^N\Phi(\lambda_i,\mu_{N-k})}\nonumber\\
		&\overline Q_{N-k}^N(\diff\lambda_1,\dots,\diff\lambda_{N-k})\diff\lambda_{N-k+1}\dots\diff\lambda_N,
	\end{align}
	
	where $\mu_{N-k}$ is determined by the variables $\lambda_1,\dots,\lambda_{N-k}$ we are integrating against.
	%For $k=1$ this inequality can be thought of as ``If I pick the largest EV lambda1 as some value between x and M, what is the probability of the other EVs being ...? basically Q_N-k^N(...) but you also have to account for the fact that they are repellent -> include this Phi'' - zeichnung! 
	
	We write $B(\rho,\delta)$ for the open ball with radius $\delta$ in $\mathcal P(\mathbb R)$ with center $\rho$ and let $B_M(\rho,\delta)=B(\rho,\delta)\cap\mathcal P([-M,M])$. Noting that $|\lambda_i-\lambda_j|\leq 2M$ on the domain of integration and $e^{(N-k)\Phi(\lambda_i,\mu_{N-k})}\leq (2M)^{N-k}$ we get the following upper bound on \ref{eq:upperbound1forQN}:
	
	\begin{align}\label{eq:upperbound2forQN}
		C_N^k\frac{N!}{(N-k)!}(2M)^{k(k-1)/2}\Big\{
			\Big(
				&\int_x^M e^{(N-k)\sup_{\mu\in B_M(\mu_{sc},\delta)}\Phi(x,\mu)}\diff x
			\Big)^k \nonumber\\
			+&(2M)^{N-k}\overline Q_{N-k}^N(\mu_{N-k}\notin B(\mu_{sc},\delta))
		\Big\}.
	\end{align}
	
	This may be seen as a crude upper bound on some case detection whether or not $\mu_{N-k}$ is in some $\delta$-neighbourhood of the semicircle law $\mu_{sc}$. Intuitively the probability of the second case occuring should vanish as $N\rightarrow\infty$ and this can be made precise by showing that $\mu_{N-k}$ under $\overline Q_{N-k}^N$ satisfies the same LDP as under $\overline Q_{N-k}$ (where it is already known to vanish by \ref{thm:LDPforWSCL}). To do so we observe that for all Lipschitz functions $h:\mathbb R \rightarrow\mathbb R$ of norm at most $1$ and $N\geq 2k$ we have the following inequality:
	
	\begin{equation}\label{eq:lipschitzRescalingBoundedByEV}
		\left|(N-k)^{-1}\sum_{i=1}^{N-k}h\left(\sqrt{1-kN^{-1}}\lambda_i\right)-h(\lambda_i)\right|\leq cN^{-1}\max_{i=k}^{N-k}|\lambda_i|,
	\end{equation}

	for some $c>0$ independent of $N$ and $k$. 
	
	The measure $\overline Q_{N-k}^N$ can be thought of describing the random variable that picks $N-k$ values as described by $\overline Q_{N-k}$ and then renormalizes them so that they on the same scale as the ones that would be picked by $\overline Q_N$ (so basically picking $N-k$ values from the distribution for $N$ values).
	
	Equation \ref{eq:lipschitzRescalingBoundedByEV} lets us bound the metric $d$ from the definition of the exponential equivalence by $cN^{-1}\max_{i=k}^{N-k}|\lambda_i|$, which in turn, using inequality \ref{eq:maxEVinequality} with $M=c^{-1}N$ yields the exponential equivalence of the two measures $\overline Q_N$ and $\overline Q_{N-k}^N$ because $\Gamma_\delta$ (as in definition \ref{def:exponentialequivalence}) is of the order $\exp(-CN^3)$ for some $C>0$.
	
	Hence, the second term in \ref{eq:upperbound2forQN} is exponentially negligible for any $\delta >0$ and $M<\infty$, which, in turn, implies that
	
	\begin{align}
		Q_N(\lambda_{N-k+1}\geq x,&\max_{i=1}^N|\lambda_i|\leq M)\leq \nonumber\\
		&\limsup_{N\rightarrow\infty}\frac{1}{N}\log C_N^k + k\lim_{\delta\searrow 0}\sup_{z\in[x,M],\mu\in B_M(\mu_{sc},\delta)}\Phi(z,\mu).
	\end{align}
	
	Since $\Phi$ is upper semi-continuous (notice that $\Phi(z,\mu)=\inf_{\eta>0}\Phi_{\eta}(z,\mu)$, where $\Phi_{\eta}(z,\mu) := \int \log(\max(|z-y|,\eta))\diff\mu(y)-z^2/4$ which is continuous on $[-M,M]\times\mathcal P([-M,M])$) the term $\lim_{\delta\searrow 0}\sup_{z\in[x,M],\mu\in B_M(\mu_{sc},\delta)}\Phi(z,\mu)$ simplifies to $\sup_{z\in[x,M]}\Phi(z,\mu_{sc})$.
	
	With $\text{supp}(\mu_{sc})=[-2,2]$, the derivative $D(z):=\frac{\partial}{\partial z}\Phi(z,\mu_{sc})$ exists for $z\geq 2$ and can be shown to be $-\sqrt{z^2/4-1}\leq 0$ (see \cite{ArousLDPforWSL}, proof of lemma 2.7) using several changes of variables and the residue method. Hence $\sup_{z\in[x,M]}\Phi(z,\mu_{sc}) = \Phi(x,\mu_{sc})=-1/2-I_1(x)$.
	
	With Selberg's formula (\cite{Mehta2004random}) it can be shown that, in the LDP, the normalization constant $\lim_{N\rightarrow\infty}N^{-1}\log C_N^k = k/2$.
	
	Putting those logarithmic asymptotics together completes the proof of the upper bound.
	
	To prove the complementary lower bound we fix $y>x>r>2$ and $\delta>0$. Similarly to the steps \ref{eq:upperbound1forQN} and \ref{eq:upperbound2forQN} we get
	
	\begin{align*}
		Q_N(&\lambda_{N-k+1}\geq x)\geq \\
		&\overline Q_N(\lambda_N\in[x,y],\dots,\lambda_{N-k+1}\in[x,y]\max_{i=1}^{N-k}|\lambda_i|\leq r)\geq\\
		&KC_N^k\exp\Big(k(N-k)\inf_{z\in[x,y],\mu\in B_r(\mu_{sc},\delta)}\Phi(z,\mu)\Big)\overline Q_{N-k}^N\big(\mu_{N-k}\in B_r(\mu_{sc},\delta)\big),
	\end{align*}
	for some $K=K(x,y,k)>0$. Again, by using the LDP of $\mu_{N-k}$ under $\overline Q_{N-k}^N$ we see that $\lim_{N\rightarrow\infty}Q_{N-k}^N(\mu_{N-k}\notin B_r(\mu_{sc},\delta))=0$.
	
	Using the behaviour of $C_N^k$ we get
	
	\begin{equation*}
		\liminf_{N\rightarrow\infty}N^{-1}\log Q_N(\lambda_{N-k+1}\geq x)\geq k(2^{-1} + \inf_{z\in[x,y],\mu\in B_r(\mu_{sc},\delta)}\Phi(z,\mu)),
	\end{equation*}
	which, after letting $\delta\searrow 0$ and $y\searrow x$ (note the continuity of $\Phi$ in the used range of parameters), yields the desired result.
\end{proof}

Another result that will be used later on is Varadhan's lemma which can be viewed as the LDP equivalent of Laplace's principle which states that $\lim_{\varepsilon\rightarrow 0} \varepsilon \int_A \exp(-\phi(x)/\varepsilon)\diff x = -\inf_{x\in A} \phi(x)$ or more loosely $\int_A \exp(-\phi(x)/\varepsilon)\diff x\approx\exp(-\inf_{x\in A}\phi(x)/\varepsilon)$.
\begin{lemma}[Varadhan]
	Let $X_\varepsilon$ be random variables in a metric space $M$ that satisfy a LDP with rate $1/\varepsilon$ and rate function $J$, and $\phi:M\rightarrow\mathbb R$ be continuous and bounded from above. Then 
	
	$$\lim_{\varepsilon\rightarrow 0}\varepsilon\log\mathbb E[\exp(\phi(X_\varepsilon)/\varepsilon)]=\sup_{x\in M}\phi(x)-J(x).$$
\end{lemma}
\begin{proof}
	This proof follows \cite{scLDP}. We will show that $\limsup$ and $\liminf$ are both equal to 	$\sup_x \phi(x)-J(x)$ and start with the $\limsup$:
	To that end pick an arbitrary positive $n$. Since $\phi$ is continuous and bounded from above, there exists a finite family of closed sets $\{B_i\}_{i=1}^{m_n}$, such that $\phi(x)\leq -n$ on the complement of their union and that within each $B_i$ $\phi$ varies only by at most $1/n$.
	Using those definitions we get
	\begin{align*}
		\limsup\varepsilon\log\mathbb E[\exp(\phi(X_\varepsilon)/\varepsilon)]
		&\leq(-n)\lor\max_{i=1}^m \limsup\varepsilon\log\mathbb E[\exp(\phi(X_\varepsilon)/\varepsilon)\mathbf 1\{X_\varepsilon\in B_i\}]\\
		&\leq(-n)\lor\max_{i=1}^m\sup_{x\in B_i}\phi(x)-\inf_{x\in B_i} J(x)\\
		&\leq(-n)\lor\max_{i=1}^m\sup_{x\in B_i}\phi(x)-J(x)+1/n\\
		&\leq(-n)\lor\sup_{x\in M}\phi(x)-J(x)+1/n
	\end{align*}
	
	Letting $n\rightarrow\infty$ we get $\limsup\varepsilon\log\mathbb E[\phi(X_\varepsilon)/\varepsilon]=\sup \phi(x)-J(x)$.
	To get the $\liminf$, pick any $x\in M$ and let $B_{\delta}(x)$ be a ball of arbitrary radius $\delta$ centered at $x$.
	We have
	
	\begin{align*}
		\liminf\varepsilon\log\mathbb E[\phi(X_\varepsilon)/\varepsilon]
		&\geq\liminf\varepsilon\log\mathbb E[\phi(X_\varepsilon)/\varepsilon\mathbf 1\{X_\varepsilon\in B_\delta(x)\}]\\
		&\geq\inf_{y\in B_\delta(x)}\phi(x)-\inf_{y\in B_\delta(x)}J(x)\\
		&\geq\inf_{y\in B_\delta(x)}\phi(x)-J(x).
	\end{align*}
	Since $\delta$ was arbitrary we can let it go to zero, yielding $\inf_{y\in B_\delta(x)}\phi(x)=\phi(x)$ by continuity of $\phi$. This holds for arbitrary $x$ so we can take the supremum over $x$ which proves the claim.
\end{proof}

\section{Kac-Rice formula}
To count zeros of a ``sufficiently well-behaved'' function $f:I\rightarrow\mathbb R$ one could try to write it as some kind of convolution. To that end we define a kernel $\varphi_\varepsilon:\mathbb R\rightarrow\mathbb R$, such that

$$
\varphi_\varepsilon(x) = \begin{cases}
	\frac{1}{2\varepsilon}, &\mbox{if } |x|<\varepsilon \\
		0, &\mbox{else.}
\end{cases}$$

Clearly $\lim_{\varepsilon\rightarrow 0}\int_\mathbb{R}\varphi_\varepsilon(x)f(x)=f(0)$ and 

\begin{equation}\label{eq:KacRice1D}
	\int_I \varphi_\varepsilon(f(x))|f'(x)|\diff x = \frac{1}{2\varepsilon}\int_{x:|f(x)|<\varepsilon}|f'(x)|\diff x,
\end{equation}

for $\varepsilon$ small enough. Now let $x_0\in f^{-1}(0)$, $x_+=\min\{x>x_0: |f(x)|=\varepsilon\}$ and $x_-=\max\{x<x_0: |f(x)|=\varepsilon\}$, then we get

$$\int_{x_-}^{x_+}|f'(x)|\diff x = |f(x_+)-f(x_-)|=2\varepsilon,$$

which motivates using \ref{eq:KacRice1D} to count zeros of a function $f$ which satisfies the following definition:
\begin{definition}[Convenient functions]
	A $\mathcal C^1$ function $f:[a,b]\rightarrow\mathbb R$ is called convenient if $f(a),f(b)\neq 0$ and all zeros are nondegenerate, i.e. $f(x)=0\implies f'(x)\neq 0$.
\end{definition}

To generalize this intuition to higher dimensions it is useful to think of the derivative factor as some kind of normalisation of the ``volume'' around the root ($f$ has the same number of zeros as $cf$ for $c\in\mathbb R\setminus\{0\}$). Thus, if we want to count critical points (i.e. substitute $f$ by $\nabla f$ and $f'$ by $\nabla^2 f$ in the above considerations), it is natural to take the determinant of the Hessian $\nabla^2 f$. Indeed this turns out to be true; however, going back to our rescaled Hamiltonian $f$, as defined in \ref{sec:GRF} we claim that

\begin{equation}\label{eq:KacRiceSN}
	\mathbb E Crt_N^f = \int_{S^{N-1}}\mathbb E[|\det\nabla^2f(\sigma)|\mathbf 1\{f(\sigma)\in B\}|\nabla f(\sigma)=0]\diff\mathbb P(\nabla f(\sigma)=0)\diff\sigma.
\end{equation}

While the above considerations motivate the choice of this integrand, it still remains to be checked what kind of criteria we have to impose on a random $f$ for this formula to hold. We will formulate these conditions in the following lemma:

\begin{theorem}[Kac-Rice formula for centered Gaussian fields]\label{thm:KacRice}
	Let $f$ be a centered Gaussian field on $S^{N-1}$ and let $\mathcal A=(U_\alpha,\Psi_\alpha)_{\alpha\in I}$ be a finite atlas on $S^{N-1}$. Set $f^\alpha = f\circ\Psi^{-1}_\alpha$ and denote the derivatives analogously to \ref{sec:GRF}. Assume that for all $\alpha\in I$ and all $x,y\in \Psi_\alpha(U_\alpha)$ the joint distribution of $(f_i^\alpha(x),f_{ij}^\alpha(x))_{1\leq i,j<N}$ is nondegenerate, and 
	
	$$\max_{i,j}|\text{Var}(f_{ij}^\alpha(x)+\text{Var}(f_{ij}^\alpha(y)-2\text{Cov}(f_{ij}^\alpha(x),f_{ij}^\alpha(y))|\leq K_\alpha|\log|x-y||^{-1-\beta}$$
	
	for some $K_\alpha,\beta>0$. Then equation \ref{eq:KacRiceSN} holds.
\end{theorem}

A proof can be found in \cite{Adler07}.








